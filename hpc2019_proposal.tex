\documentclass[12pt]{article}

%% FONTS
%% To get the default sans serif font in latex, uncomment following line:
 \renewcommand*\familydefault{\sfdefault}
%%
%% to get Arial font as the sans serif font, uncomment following line:
%% \renewcommand{\sfdefault}{phv} % phv is the Arial font
%%
%% to get Helvetica font as the sans serif font, uncomment following line:
% \usepackage{helvet}
\usepackage[small,bf,up]{caption}
\renewcommand{\captionfont}{\footnotesize}
\usepackage[left=1in,right=1in,top=1in,bottom=1in]{geometry}
\usepackage{graphics,epsfig,graphicx,float,subfigure,color}
\usepackage{amsmath,amssymb,amsbsy,amsfonts,amsthm}
\usepackage{url}
\usepackage{boxedminipage}
\usepackage[sf,bf,tiny]{titlesec}
 \usepackage[plainpages=false, colorlinks=true,
   citecolor=blue, filecolor=blue, linkcolor=blue,
   urlcolor=blue]{hyperref}
\usepackage{enumitem}
\usepackage{verbatim}
\usepackage{tikz,pgfplots}

\newcommand{\todo}[1]{\textcolor{red}{#1}}
% see documentation for titlesec package
% \titleformat{\section}{\large \sffamily \bfseries}
\titlelabel{\thetitle.\,\,\,}

\newcommand{\bs}{\boldsymbol}
\newcommand{\alert}[1]{\textcolor{red}{#1}}
\setlength{\emergencystretch}{20pt}

\begin{document}

\begin{center}
  \vspace*{-2cm}
{\small MATH-GA 2012.001 and CSCI-GA 2945.001, Georg Stadler \&
  Dhairya Malhotra (NYU Courant)}
\end{center}
\vspace*{.5cm}
\begin{center}
\large \textbf{%%
Spring 2019: Advanced Topics in Numerical Analysis: \\
High Performance Computing \\
Final Project Proposal }
\end{center}

\vspace*{0.4cm}
\begin{center}
\large \textbf{Anya Katsevich and Terrence Alsup}
\end{center}

\par Anya and I will work on parallelizing Kinetic Monte Carlo (KMC).  KMC is a model for the movement of atoms on the surface of the crystal.  In 2D we consider a lattice $\Omega = \{1,\ldots,L\} \times \{1,\ldots,L\}$.  At each site $\vec{\ell} \in \Omega$ on the lattice we have atoms stacked on top of each other.  The number of atoms at each site $\vec{\ell}$ will be denoted by the height of the stack $h_{\vec{\ell}}$.  The surface of the crystal is then just the atoms at the top of each stack.  KMC simulates a Markov jump process on the vector $(h_{\vec{\ell}})_{\vec{\ell} \in \Omega}$ by randomly choosing one atom on the surface to jump on top of one of the neighboring stacks.  The rates of this process are governed by height difference between two neighboring stacks.  In serial there is only 1 jump at a time and all other atoms, even those far away, must wait.\\

\par To parallelize KMC we will divide the lattice into sections and then independently run Markov jump processes on each section.  After every process has run for a while we will need to synchronize the sections since the jump times are random.  We will also need to handle potential conflicts on the boundaries of the sections where atoms may be jumping across.  This could potentially be handled with a ghost layer.  We plan to transfer the computation to a GPU for the parallelization.  There is a scaling limit whenever we rescale both time $t$ and the size of the lattice $L$, which is a PDE governing the dynamics of the surface of the crystal.  To test our implementation we can do the rescaling and compare the error with the computed solution of the PDE.

\end{document}

